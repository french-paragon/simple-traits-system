\documentclass[a4paper,10pt,twoside,twocolumn]{article}

\usepackage{ucs}
\usepackage[utf8]{inputenc}
\usepackage{amsmath}
\usepackage{amsfonts}
\usepackage{amssymb}
\usepackage[francais]{babel}
\usepackage{fontenc}
\usepackage{graphicx}

\usepackage[dvips]{hyperref}

\author{Laurent "Paragon" Valentin Jospin}
\title{Simple Traits System}
\date{25.02.2017}

\begin{document}

  \maketitle
 
 \section{Introduction}
 
 Ce document est le document de référence du Simple Traits System, ou système de traits simple abrégé sts. Il s'agit d'un système générique pour le jeu de rôle sur table optimisé pour être rapide à mettre en oeuvre, donner une grande liberté dans la détermination des personnages tout en recréant une ambiance proche des systèmes ``classiques'' tel que le d20 system ou le système d100.\\
 \\
 Ce document est à destination des maîtres du jeu désirant utiliser le système générique, ainsi que des auteurs et éditeurs qui pourraient être intéréssé à employer le système en lien avec leurs univers. Il est assumé que le lecteur a déjà une idée assez claire de comment fonctionne un jeu de rôle sur table.
 
 \section{License et utilisation}
 
 Le système de jeu en tant que tel est placé sous license Creative Commons Attribution (CC-BY), vous êtes libres de partager ou transformer le présent texte, y compris pour une utilisation commerciale, tant que vous citez l'auteur original.\\
 \\
 Les illustration utilisées dans ce document sont placées sous license Creations Common Attribution NonCommercial ShareAlike (CC-BY-NC-SA), vous êtes libre de partager et remixer ces images tant que ce n'est pas pour une utilisation commerciale, que vous citez l'auteur original ainsi que tout les auteurs intermédiaires et repartagez votre travail avec la même license (CC-BY-NC-SA).\\
 \\
 Pour plus d'information sur les licenses vous pouvez consulter creativecommons.org.
 
\end{document}
